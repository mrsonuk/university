\documentclass[11pt, a4paper]{article}

\usepackage[utf8]{inputenc}
\usepackage{pgffor}  %foreach
\usepackage{xstring} %string methods
\usepackage{fancyhdr}
\usepackage{ amssymb, amsmath, amsthm, dsfont }
\usepackage{ marvosym }
\usepackage{ stmaryrd }
%\usepackage[right = 1cm, top = 2cm]{geometry}
\usepackage[width = 18cm, top = 2cm, bottom = 3cm]{geometry}
\usepackage[pdf]{graphviz}
\usepackage{dot2texi}
\usepackage{tikz}
\usepackage{float}
\usetikzlibrary{shapes.geometric,shapes.misc}
\usetikzlibrary{arrows.meta,arrows}
\usepackage{tikz-qtree,tikz-qtree-compat}
\usepackage{forest}
\usepackage{pdflscape}
%--- Own commands

\newcommand{\anf}[1]{``#1''}
\newcommand{\punkt}[2]{\hfill\begin{small} $[#1$ \textnormal{Punkt#2}$]$\end{small}}
\newcommand{\punkttabelle}[1]{$\hspace{0.5cm} /$ #1}
\newcommand{\abstractTabular}[3]
{
	\begin{center}
		\begin{normalsize}
			\begin{tabular}{#1}
				Aufgabe & #2 & $\Sigma$ \\
				\hline
				Punkte &  #3 \\
			\end{tabular}
		\end{normalsize}
	\end{center}
}

% -------------

\newcommand{\myTitleString}		{}
\newcommand{\myAuthorString}	{}
\newcommand{\mySubTitleString}	{}
\newcommand{\myDateString}		{}

\newcommand{\myTitle}		[1]{\renewcommand{\myTitleString}		{#1}}
\newcommand{\mySubTitle}	[1]{\renewcommand{\mySubTitleString}	{#1}}
\newcommand{\myAuthor}		[1]{\renewcommand{\myAuthorString}		{#1}}
\newcommand{\myDate}		[1]{\renewcommand{\myDateString}		{#1}}

\newcommand{\makeMyTitle}
{
	\thispagestyle{fancy} 					%eigener Seitenstil
	\fancyhf{} 								%alle Kopf- und Fußzeilenfelder bereinigen
	\fancyhead[L]							%Kopfzeile links
	{
		\begin{tabular}{l}
			\myTitleString
			\\ \mySubTitleString
			\\ \myDateString
		\end{tabular}
	}
	\fancyhead[C]
	{}	%zentrierte Kopfzeile
	\fancyhead[R]{\myAuthorString}	   		%Kopfzeile rechts
	\fancyfoot[C]{\thepage} 				%Seitennummer
	\text{}
}

% -------------

\renewcommand{\v}{\vee}
\newcommand{\n}{\wedge}
\newcommand{\xor}{\oplus}
\newcommand{\nmodels}{\nvDash}
\newcommand{\interp}{\mathfrak{I}}
\newlength\tindent
\setlength{\tindent}{\parindent}
\setlength{\parindent}{0pt}
\renewcommand{\indent}{\hspace*{\tindent}}


\begin{document}
\myTitle{MaLo}
\mySubTitle{Übung 12 - Gruppe A}
\myDate{SS 18}
\myAuthor
{
	\begin{tabular}{l l}
		378625, & Felix Dittmann \\
		380678, & Clemens Rüttermann \\
		381852, & Richard Zameitat \\
	\end{tabular}
}
\makeMyTitle

\section*{Aufgabe 1}
Zwecks Widerspruch nehmen wir an, es gibt eine maximale Bisimulation $Z$ bzgl. $\subseteq$, sodass $Z' \subseteq Z$ nicht für alle Bisimulationen $Z'$.
Das heißt, dass mindestens eine Bisimulation $Y$ existiert, für welche gilt $Y \nsubseteq Z$.
Das würde bedeuten, dass $Y$ Elemente enthält, welche nicht in $Z$ enthalten sind.
Dies steht allerdings im Wiederspruch zu der Annahme, dass $Z$ maximal ist.



\section*{Aufgabe 2}
\[
    Z = \{ (1,a), (2,b), (3,c), (4,c), (5,b), (6,d) \}
\]

$a$:\\
    Für den Knoten $a$ existiert keine weitere mögliche Zuordnung als $1$, da dieser Knoten einen $Q$- und einen $P$-Nachfolger haben müsste und selbst keine Beschriftung haben dürfte.
    Dies trifft aber nur auf $1$ zu.
\\

$b,c$:\\
    Für $b$ und $c$ existieren ebenfalls keine weiteren Zuordnungen, da alle $P$-, bzw. $Q$-Knoten bereits zugeordnet wurden.
\\

$d$:\\
    Für $d$ gibt es keine weitere Zuordnung, da der Knoten selbst nicht beschriftet sein dürfte und seine Nachfolger auch nicht.

\section*{Aufgabe 3}

\subsection*{a)}
Sei $\mathcal{K}_1 = (\{1,2,3\}, \{(1,2), (2,3), (3,1)\}, \O)$ die Kripkestruktur eines Kreises der Länger 3.
Dann die Kripkestruktur $\mathcal{K}_2 = (\{a_1, a_2, a_3, \cdots\}, \{(a_1,a_2), (a_2,a_3), \cdots\}, \O)$
eines Pfades unendlicher Länge mit der Modallogik nicht von $\mathcal{K}_1$ unterscheidbar,
da die Modallogik keine Aussagen zur Identität oder eines Vorgängers des aktuellen Knotens treffen kann und der unendlich lange Pfad somit die Anforderungen an einen Kreis der Länge 3 (mindestens ein Nachfolger mit Nachfolger und eine unendlich Lange Kette von Nachfolgern) erfüllt.

Somit ist die Aussage \textit{"Der Aktuele Knoten liegt auf einem Kreis der Länge 3"} nicht in der Modallogik formulierbar.

\subsection*{b)}
Es kann in der Modallogik nicht spezifiziert werden, dass ein Knoten den aktuellen Knoten als Nachfolger hat, da die Identität von Knoten nicht verwendbar ist und weithin auch nur die Nachfolge-Relation zur Verfügung steht, auf Vorgänger also kein Bezug genommen werden kann.

Sei $\mathcal{K}_3$ eine Kripkrestruktur, die einen Knoten enthät, der mit P beschriftet ist und einen weiteren Knoten als Nachfolger hat.
Jener Nachfolger sei der aktuelle Knoten und wiederrum mit dem mit P bechrifteten Knoten verbunden.
Diese Struktur erfüllt die Bedingung, welche mit der Modallogik geprüft werden soll.
Eine Formel der Modallogik, welche diese Struktur akzeptiert kann lauten $\diamond P \diamond 1$.

Allerdings wird die Kripkestruktur von dieser Formel akzeptiert, welche ähnlich aufgebaut ist, hierbei aber auf die Verbindung vom aktuellen zum mit P markierten Knoten verzichtet, nicht mehr erkannt, obwohl auch diese die Bedingung erfüllt.
Dies zeit der Problem bei der Verwendung der Modallogik zur spezifikation eines Vorgängers.

Somit ist auch die Aussage \textit{"Es gibt einen Knoten, der mit  beschriftet ist, von dem aus eine Kante zu dem aktuellen Knoten besteht"} nicht in der Modallogik formulierbar.

\subsection*{c)}
\[
    \varphi_c := \diamond ( \diamond P \n \square \overline{Q} ) \n \square ( \square \overline{P} \n \exists Q) \n 0 \n \square \square \neg 1
\]

\section*{Aufgabe 4}

\subsection*{a)}
$X \v (Y \n (Z \rightarrow V)) \equiv\\$
$(V \v X \v \neg Z) \n (X \v Y)\\$

Da die Formel nun in KNF ist aber nicht jede Disjunktion höchstens ein positives Literal hat, ist diese Formel nicht äquivalent zu einer Horn-Formel.



\subsection*{b)}

\subsection*{c)}

\subsection*{d)}

\subsection*{e)}

\end{document}
