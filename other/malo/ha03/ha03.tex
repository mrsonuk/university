\documentclass[11pt, a4paper]{article}

\usepackage[utf8]{inputenc}
\usepackage{pgffor}  %foreach
\usepackage{xstring} %string methods
\usepackage{fancyhdr}
\usepackage{ amssymb, amsmath, amsthm, dsfont }
\usepackage{ marvosym }
\usepackage{ stmaryrd }
%\usepackage[right = 1cm, top = 2cm]{geometry}
\usepackage[width = 18cm, top = 2cm, bottom = 3cm]{geometry}
\usepackage[pdf]{graphviz}
\usepackage{dot2texi}
\usepackage{tikz}
\usepackage{float}
\usetikzlibrary{shapes.geometric,shapes.misc}
\usepackage{tikz-qtree,tikz-qtree-compat}
%--- Own commands

\newcommand{\anf}[1]{``#1''}
\newcommand{\punkt}[2]{\hfill\begin{small} $[#1$ \textnormal{Punkt#2}$]$\end{small}}
\newcommand{\punkttabelle}[1]{$\hspace{0.5cm} /$ #1}
\newcommand{\abstractTabular}[3]
{
	\begin{center}
		\begin{normalsize}
			\begin{tabular}{#1}
				Aufgabe & #2 & $\Sigma$ \\
				\hline
				Punkte &  #3 \\
			\end{tabular}
		\end{normalsize}
	\end{center}
}

% -------------

\newcommand{\myTitleString}		{}
\newcommand{\myAuthorString}	{}
\newcommand{\mySubTitleString}	{}
\newcommand{\myDateString}		{}

\newcommand{\myTitle}		[1]{\renewcommand{\myTitleString}		{#1}}
\newcommand{\mySubTitle}	[1]{\renewcommand{\mySubTitleString}	{#1}}
\newcommand{\myAuthor}		[1]{\renewcommand{\myAuthorString}		{#1}}
\newcommand{\myDate}		[1]{\renewcommand{\myDateString}		{#1}}

\newcommand{\makeMyTitle}
{
	\thispagestyle{fancy} 					%eigener Seitenstil
	\fancyhf{} 								%alle Kopf- und Fußzeilenfelder bereinigen
	\fancyhead[L]							%Kopfzeile links
	{
		\begin{tabular}{l}
			\myTitleString
			\\ \mySubTitleString
			\\ \myDateString
		\end{tabular}
	}
	\fancyhead[C]
	{}	%zentrierte Kopfzeile
	\fancyhead[R]{\myAuthorString}	   		%Kopfzeile rechts
	\fancyfoot[C]{\thepage} 				%Seitennummer
	\text{}
}

% -------------

\renewcommand{\v}{\vee}
\newcommand{\n}{\wedge}
\newcommand{\xor}{\oplus}
\newcommand{\nmodels}{\nvDash}
\newcommand{\interp}{\mathfrak{I}}
\newlength\tindent
\setlength{\tindent}{\parindent}
\setlength{\parindent}{0pt}
\renewcommand{\indent}{\hspace*{\tindent}}

\begin{document}
\myTitle{MaLo}
\mySubTitle{Übung 03 - Gruppe A}
\myDate{SS 18}
\myAuthor
{
	\begin{tabular}{l l}
		381852, & Richard Zameitat \\
		378625, & Felix Dittmann \\
		380678, & Clemens Rüttermann \\
	\end{tabular}
}
\makeMyTitle

\section*{Aufgabe 1}
Sei $u\subseteq \mathbb{N}$. Dann definieren wir die Aussagenvariable $X_u$ derart, dass $X_u\mapsto 1$ bedeutet, dass die Menge $u$ als klein kategorisiert wird, und $X_u\mapsto 0$, dass $u$ als groß kategorisiert wird.\newline
Wir können mithilfe von $X_u$ eine Formelmenge $\Phi$ erstellen, welche die Kategorisierung der Teilmengen der natürlichen Zahlen beschreibt. Dazu erstellen wir für die Bedingungen Teilmengen von $\Phi$:\newline
a) Diese Bedingung wird durch die Definition von $X_u$ erfüllt, da $X_u$ nicht sowohl auf $1$ als auch auf $0$ abgebildet werden kann.\newline
b) $\Phi_b = \{X_u: u\subseteq \mathbb{N}$ und u ist endlich$\}$\newline
c) $\Phi_c = \{X_u\wedge X_v\to X_{u\cup v}: u,v\subseteq \mathbb{N} \}$\newline
d) $\Phi_d = \{X_u\wedge X_v: u\subseteq \mathbb{N}, v\subseteq u \}$\newline
e) $\Phi_e = \{X_u \leftrightarrow \neg X_v: u\subseteq \mathbb{N}, v=\mathbb{N}\setminus u \}$\newline
Somit ergibt sich $\Phi=\Phi_b \cup \Phi_c \cup \Phi_d \cup \Phi_e$.\newline
Wenn $\Phi$ erfüllbar ist, können die Teilmengen der natürlichen Zahlen mit den entsprechenden Bedingungen a) bis e) in "groß" und "klein" kategorisiert werden.\newline
Dank des Kompaktheitssatzes wissen wir, dass $\Phi$ erfüllbar ist, wenn alle endlichen Teilmengen von $\Phi$ erfüllbar sind.\newline
Um dies zu zeigen, betrachten wir eine beliebige Teilmenge $\Phi_0$. Da $\Phi_0$ eine endliche Teilmenge ist, bedeutet dies, dass $\Phi_0$ nur endlich viele Formeln und somit nur endlich viele verschiedene Aussagenvariablen $X_u$ enthält. Somit betrachten wir auch nur endlich viele Mengen $u$ und auch nur endlich viele $u$, die eine endliche Teilmenge sind.\newline
Somit können wir für jedes $\Phi_0$ ein $n\in \mathbb{N}$ finden, welches nicht in den endlichen Teilmengen von $\Phi_0$ liegt. Mithilfe von $n$ können wir schließlich eine Interpretation $\mathfrak{I}_n$ bilden, die $X\mapsto 1$ genau dann, wenn $n$ nicht in $u$ enthalten ist, und $X\mapsto 0$ genau dann, wenn $n$ in $u$ enthalten ist.\newline
Da $n$ in keiner endlichen Teilmenge $u$ von $\Phi_0$ enthalten ist, ist jede endliche Teilmenge somit klein (b)). Da $n$ auch in keiner kleinen Menge enthalten ist, ist $n$ auch nicht in deren Vereinigung (c)) oder einer Teilmenge von $u$ (d)) enthalten. Schließlich muss noch das Komplement einer Menge, die $n$ nicht enthält, $n$ enthalten (e)). Somit ist $\mathfrak{I}_n$ ein Modell von $\Phi_0$.\newline
Da sich ein derartiges $n$ für jedes $\Phi_0$ finden lässt, können wir auch eine entsprechende Interpretation $\mathfrak{I}_n$ für jedes $\Phi_0$ bilden. Somit ist jedes $\Phi_0$ erfüllbar und mit dem Kompaktheitssatz ist somit auch $\Phi$ erfüllbar.\newline
Dadurch haben wir gezeigt, dass sich jede Teilmenge der natürlichen Zahlen als "groß" oder "klein" kategorisieren lässt.


\section*{Aufgabe 2}
\subsection*{(a)}
siehe beschriebene Blätter

\subsection*{(b)}
siehe beschriebene Blätter

\section*{Aufgabe 3}
\subsection*{(a)}
Das Doppelresolutionskalkül ist vollständig, da es sich bei dem Resolutionskalkül um einen Spezialfall des Doppelresolutionskalküls, in welchem immer $Y=Z$ gilt. Somit lassen sich alle Aussagen durch das Doppelresolutionskalkül ableiten, die sich auch durch das Resolutionskalkül ableiten lassen. Da das Resolutionskalkül vollständig ist, ist damit auch das Doppelresolutionskalkül vollständig.

\subsection*{(b)}
Das Doppelresolutionskalkül ist nicht korrekt:\newline
Gegenbeispiel: $C_1=\{X,Y\},C_2=\{\neg X,\neg Y\}$\newline
Laut Doppelresolutionskalkül existiert dann eine Resolvente $C=\{ \square \}$, welche Unerfüllbarkeit andeutet. Allerdings besitzt die repräsentierte Formel $\phi: X\lor Y\land \neg X \lor \neg Y$ ein Modell $\mathfrak{I}: X\mapsto 1, Y\mapsto 0$ und ist somit erfüllbar. Da sich mit dem Doppelresolutionskalkül eine falsche Aussage ableiten lässt, ist es nicht korrekt.

\section*{Aufgabe 4}
\subsection*{(a)}
Behauptung: Jede Klauselmenge, in der jede Klausel mindestens ein positives Literal enthält, lässt sich durch die Interpretation $\mathfrak{I}$ erfüllen, die alle Literale auf 1 abbildet.\newline
Beweis: Nehmen wir an, dass die Behauptung falsch sei. Somit existiert eine Klauselmenge $K$, in der jede Klausel mindestens ein positives Literal enthält, die durch $\mathfrak{I}$ aber nicht erfüllt ist. Damit existiert eine Klausel, $C\in K$, die durch $\mathfrak{I}$ nicht erfüllt wird. Da $\mathfrak{I}$ alle Literale von $C$ auf 1 abbildet, kann es somit kein positives Literal in $C$ geben. Dies ist ein Widerspruch zur Voraussetzung, dass jede Klausel mindestens ein positives Literal enthält. Somit ist die Behauptung wahr.

\subsection*{(b)}
Zu zeigen: Das nagative Resolutionsklakül ist korrekt, also $\square \in NegRes^*(K) \implies \text{ Klauselmenge } K \text{ ist unerfüllbar}$.

%% Ansatz 1
%Das negative Resolutionskalkül erlaubt die Resolventenbildung für zwei Klauseln $C_1$ und $C_2$ nur, wenn mindestens eine der beiden Klauseln nur negative Literale enthält.
%Ohne Beschränkung der Allgemeinheit wird angenommen, dass die Klausel, welche nur negative Literale enthält $C_1$ ist, also $C_1 = X_1 \vee \cdots \vee X_n$, mit allen $X_i ist negatives Literal$.
%
%---
%
%% Ansatz 2
%Wir nehmen an, dass das negative Resolutionskalkül nicht korrekt ist.
%Dann muss gelten, dass es mindestens Formelmenge $K$ gibt, für die $\square \in NegRes^*(K)$, aber $K$ ist erfüllbar.
%
%---

% Ansatz 3 ... jetzt aber!
Wir wissen, dass das Resolutionskalkül korrekt ist.
Das negative Resolutionskalkül enhtält lediglich eine zusätzliche Anforderung zur Resolventenbildung.
Daher kann im negativen Resolutionskalkül eine Resolvente für zwei Klauseln nur gebildet werden, wenn dies auch im Resolutionskalkül möglich ist.
Somit lassen sich mit dem negativen Resolutionskalkül auch nur Klauselmengen auf $\square$ reduzieren, für die das auch im Resolutionskalkül möglich ist, also gilt eine Klauselmenge im negativen Resolutionskalkül nur dann als unerfüllbar, wenn dies auch im Resolutionskalkül gilt.
Da das Resolutionskalkül korrekt ist, ist also auch das negative Resolutionskalkül korrekt.

\subsection*{(c)}
% Skript Seite 27 bzw. 31 und folgende
Sei $K$ unerfüllbar.
Es gibt also eine endliche Teilmenge $K_0 \subseteq K$ die unerfüllbar ist (Kompaktheitssatz).
Dann existiert ein $n \in \mathbb{N}$, so dass in $K_0$ nicht mehr als die Variablen $X_0, \cdots, X_{n-1}$ enthalten sind.
Es wird gezeigt: $\square \in NegRes^*(K_0) \subseteq NegRes^*(K)$.
\\

Induktionsanfang:
Sei $n=0$. Es existieren genau zwei Klauselmengen, $\emptyset$ und $\{\square\}$ ohne Variablen.
$\emptyset$ ist erfüllbar, also muss gelten $K_0 = \{\square\}$.

Induktionsschritt:
Annahme: Alle Variablen von $K_0$ sind in $\{X_0, \cdots, X_n\}$ enthalten.
Seien $K_0^+$ und $K_0^-$ Klauselmengen:
\begin{align*}
    K_0^+ :=& \{ C \setminus \{ \neg X_n \}: X_n \notin C, C \in K_0 \},\\
    K_0^- :=& \{ C \setminus \{ X_n \}: \neg X_n \notin C, C \in K_0 \}
\end{align*}

$K_0^+$ und $K_0^-$ sind unerfüllbar, da es sonst $\mathfrak{I}:\{X_0,\cdots,X_{n-1}\} \rightarrow \{0,1\}$ mit $\llbracket K_0^+ \rrbracket^\mathfrak{I} = 1$ gäbe.
Sei $\mathfrak{I}(X_n) = 1$. Dann gilt $\llbracket K_0 \rrbracket^\mathfrak{I} = 1$:\\
Für eine beliebige Klausel $C \in K_0$ gilt, dass diese nun $X_n$ enthalten kann und wegen $\mathfrak{I}(X_n)$ gilt $\llbracket C \rrbracket^\mathfrak{I} = 1$.
Ist $X_n$ nicht in $C$ enthalten, dann ist $C \setminus \{ \neg X_n \} \in K_0^+$ und durch $\llbracket K_0^+ \rrbracket^\mathfrak{I}$ gilt zwangsweise $\llbracket C \setminus \{ \neg X_n \} \rrbracket^\mathfrak{I}$.
Hieraus folgt $\llbracket C \rrbracket^\mathfrak{I} = 1$ und somit auch $\llbracket K_0 \rrbracket^\mathfrak{I} = 1$. $\;_\text{\LARGE\Lightning}$\\
Dies ist wiedersprüchlich zur Unerfüllbarkeit von $K_0$.
Analog zu obigem, durch setzen von $\mathfrak{I}(X_n) = 0$, ist auch die Unerfüllbarkeit von $K_0^-$ zeigbar.

Somit folgt durch die Induktion, dass $\square \in NegRes^*(K_0^+)$ und $\square \in NegRes^*(K_0^-)$.
Es existieren also Klauseln $C_1,\cdots,C_m$ so dass $C_m = \square$ und, für $i=1,\cdots,m$, gilt $C_i$ ist Resolvente von $C_j,C_k$ für $j,k < i$ oder $C_i \in K_0^+$.\\
Sollten Klauseln $C_i$ durch Streichen von $\neg X_n$ entstanden sein, so entsteht durch Weidereinfügen von $\neg X_n$ eine Klauselfolge $C_1',\cdots,C_m'$, die $\{ \neg X_n \} \in NegRes^*(K_0)$ zeigt.
Sollte das nicht der Fall gewesen sein, dann sind $C_1,\cdots,C_m$ in $NegRes^*(K_0)$ und damit $\square \in NegRes^*(K_0)$.\\
Für $K_0^-$ kann wieder analog verfahren werden und es folgt aus $\square \in NegRes^*(K_0^-)$, dass entweder $\{X_n\} \in NegRes^*(K_0)$ oder $\square \in NegRes^*(K_0)$.

Es folgt, dass $\square \in NegRes^*(K_0)$.


\section*{Aufgabe 5}
\subsection*{a)}
$\Gamma, \phi, \phi \rightarrow \psi \Rightarrow \Delta, \psi$\\
------------------------------------------------------------------------$(\rightarrow \Rightarrow)$\\
$(+)\quad\Gamma, \psi \Rightarrow \Delta, \psi, \phi \quad \Gamma, \phi, \psi \Rightarrow \Delta, \psi, \phi(+)$

Diese Ausdrücke sind Axiome

\subsection*{b)}
$\Gamma, \phi \v \psi \Rightarrow \Delta, \phi \n \psi$\\
------------------------------------------------------------------------$(\v \Rightarrow)$\\
$\quad\Gamma, \phi \Rightarrow \Delta, \phi \n \psi \quad \Gamma, \psi \Rightarrow \Delta, \phi \n \psi$\\
------------------------------------------------------------------------$(\Rightarrow \n)$\\
$\quad\Gamma, \phi \Rightarrow \Delta, \phi \quad \Gamma, \phi \Rightarrow \Delta, \psi \quad \Gamma, \psi \Rightarrow \Delta, \phi \quad \Gamma, \phi \Rightarrow \Delta, \psi$

Alle Blätter sind Axiome


\subsection*{c)}
$\Gamma, \phi, \neg \phi \Rightarrow \emptyset$\\
------------------------------------$(\neg \Rightarrow)$\\
$\Gamma, \phi \Rightarrow \phi$\\
$\quad (+)$

\subsection*{d)}
$\Gamma \Rightarrow \Delta, \phi, \neg \phi$\\
------------------------------------$(\Rightarrow \neg)$\\
$\quad\Gamma, \phi \Rightarrow \Delta, \phi$

Dies ist ein Axiom

\subsection*{e)}
$\Gamma, \phi \n \psi \Rightarrow \Delta, \phi, \psi$\\
------------------------------------$(\n \Rightarrow)$\\
$\Gamma, \phi, \psi \Rightarrow \Delta, \phi, \psi$
















\end{document}
