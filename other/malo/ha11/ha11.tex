\documentclass[11pt, a4paper]{article}

\usepackage[utf8]{inputenc}
\usepackage{pgffor}  %foreach
\usepackage{xstring} %string methods
\usepackage{fancyhdr}
\usepackage{ amssymb, amsmath, amsthm, dsfont }
\usepackage{ marvosym }
\usepackage{ stmaryrd }
%\usepackage[right = 1cm, top = 2cm]{geometry}
\usepackage[width = 18cm, top = 2cm, bottom = 3cm]{geometry}
\usepackage[pdf]{graphviz}
\usepackage{dot2texi}
\usepackage{tikz}
\usepackage{float}
\usetikzlibrary{shapes.geometric,shapes.misc}
\usetikzlibrary{arrows.meta,arrows}
\usepackage{tikz-qtree,tikz-qtree-compat}
\usepackage{forest}
\usepackage{pdflscape}
%--- Own commands

\newcommand{\anf}[1]{``#1''}
\newcommand{\punkt}[2]{\hfill\begin{small} $[#1$ \textnormal{Punkt#2}$]$\end{small}}
\newcommand{\punkttabelle}[1]{$\hspace{0.5cm} /$ #1}
\newcommand{\abstractTabular}[3]
{
	\begin{center}
		\begin{normalsize}
			\begin{tabular}{#1}
				Aufgabe & #2 & $\Sigma$ \\
				\hline
				Punkte &  #3 \\
			\end{tabular}
		\end{normalsize}
	\end{center}
}

% -------------

\newcommand{\myTitleString}		{}
\newcommand{\myAuthorString}	{}
\newcommand{\mySubTitleString}	{}
\newcommand{\myDateString}		{}

\newcommand{\myTitle}		[1]{\renewcommand{\myTitleString}		{#1}}
\newcommand{\mySubTitle}	[1]{\renewcommand{\mySubTitleString}	{#1}}
\newcommand{\myAuthor}		[1]{\renewcommand{\myAuthorString}		{#1}}
\newcommand{\myDate}		[1]{\renewcommand{\myDateString}		{#1}}

\newcommand{\makeMyTitle}
{
	\thispagestyle{fancy} 					%eigener Seitenstil
	\fancyhf{} 								%alle Kopf- und Fußzeilenfelder bereinigen
	\fancyhead[L]							%Kopfzeile links
	{
		\begin{tabular}{l}
			\myTitleString
			\\ \mySubTitleString
			\\ \myDateString
		\end{tabular}
	}
	\fancyhead[C]
	{}	%zentrierte Kopfzeile
	\fancyhead[R]{\myAuthorString}	   		%Kopfzeile rechts
	\fancyfoot[C]{\thepage} 				%Seitennummer
	\text{}
}

% -------------

\renewcommand{\v}{\vee}
\newcommand{\n}{\wedge}
\newcommand{\xor}{\oplus}
\newcommand{\nmodels}{\nvDash}
\newcommand{\interp}{\mathfrak{I}}
\newlength\tindent
\setlength{\tindent}{\parindent}
\setlength{\parindent}{0pt}
\renewcommand{\indent}{\hspace*{\tindent}}


\begin{document}
\myTitle{MaLo}
\mySubTitle{Übung 11 - Gruppe A}
\myDate{SS 18}
\myAuthor
{
	\begin{tabular}{l l}
		378625, & Felix Dittmann \\
		380678, & Clemens Rüttermann \\
		381852, & Richard Zameitat \\
	\end{tabular}
}
\makeMyTitle

\section*{Aufgabe 1}
\textbf{Unendliche Sterne sind nicht endlich axiomatisierbar}\\
\begin{align*}
    \varphi_{\geq n} &:= \exists x_1 \cdots \exists x_n (\bigwedge_{i \neq j} x_i \neq x_j) \\
    \Phi_\infty &:= \{ \varphi_{\geq n} | n \in \mathbb{N} \} \\
    \varphi_\star &:= \exists x \forall y \forall z (Eyz \leftrightarrow (x = y \n z \neq x)) \\
    \Phi_{\infty\star} &:= \Phi_\infty \cup \{ \varphi_\star \}
\end{align*}

Angenommen, $\psi_{\infty\star}$ axiomatisiert die Klasse der unendlichen Sterne.
Dann gilt $\Phi_{\infty\star} \models \psi_{\infty\star}$.
Nach dem Kompaktheitssatz existiert ein endliches $\Phi_0 \subseteq \Phi_{\infty\star}$ mit $\Phi_0 \models \psi_{\infty\star}$ .

Sei $n$ maximal mit $\phi_{\geq n} \in \Phi_0$.
Betrache den Stern mit $n$ Knoten, dann ist dieser ein Modell von $\Phi_0$, aber nicht von $\psi_{\infty\star}$.
$\!_{\text{\Large\Lightning}}\;$
Dies steht im Widerspruch zur Annahme, also ist die Klasse der unendlichen Sterne nicht enldich axiomatisierbar.

Also können wir unendliche Sterne hier für die Entscheidbarkeit des Erfüllbarkeitsproblems ignorieren.

\null
\textbf{Erfüllung zeigen ist endlich}\\
Die Elemente der Klasse der endlichen Sterne sind, über die Anzahl der Knoten, aufzählbar, da es für jede Anzahl $n$ von Knoten nur einen passenden Stern gibt.

Da (siehe oben) die Klasse der unendlichen Sterne nicht endlich axiomatisierbar ist, kann eine einzelne FO-Formel nicht ausschließlich einen unendlichen Stern als Modell haben, sondern muss, wenn sie von einem unendlichen Stern erfüllt wird, auch noch von einem endlichen erfüllt werden.
Somit ist die Überprüfung, ob ein Stern Modell einer FO-Formel ist in endlicher Zeit durchführbar, indem die endlichen Sterne aufgezählt werden und im Fall, dass ein Stern die Formel erfüllt der Algorithmus beendet und "JA" als Ergebnis zurückgegeben wird.
Hierfür kann das Model-Checking-Verfahren verwendet werden.

\null
\textbf{Nicht-Erfüllung zu zeigen ist endlich}\\
Wir wollen überprüfen, ob für die gegebene Formel $\varphi$, $\neg\varphi$ eine Tautologie ist, also $\varphi \Rightarrow \O$ gültig ist.
Dies ist, nach dem Vollständigkeitssatz, beweisbar.

Hierfür generieren wir nacheinander alle, sich aus $\tau(\varphi)$ ergebenen, Sequenzfolgen und überprüfen, ob eine davon ein Beweis von $\varphi \Rightarrow \O$ ist.
Sollte dies der Fall sein, können wir abbrechen und "NEIN" zurückgeben.

Die Menge, der aus $\tau$ erzeugbaren Sequenzen ist endlich.
Somit sind Ableitung und Überprüfung auf Unerfüllbarkeit in endlicher Zeit durchführbar.

\null
\textbf{Parallele Anwendung beider Algotihmen}\\
Wendet man nun die Algorithmen zum Nachweis von Erfüllung und Nicht-Erfüllung parallel auf die gegebene Formel an, entscheidet jeder der Teilalgorithmen im Fall, dass die jeweilige Annahme zutrifft, in endlicher Zeit, dass dem so ist. Nach dem einer der Teil-Algorithmen beendet wurde kann auch der Gesamtalgorithmus beenden und die Rückgabe übernhemen.

Somit die das Erfüllbarkeitkeitsproblem von $FO(\tau) \; modulo \; \mathcal{K}$ in endlicher Zeit entscheidbar.


\section*{Aufgabe 2}

\subsection*{a)}
$\varphi_1: v$ hat einen Nachfolger, für dessen Nachfolger gilt, wenn sie einen P Nachfolger haben, dann haben sie auch einen Q-Nachfolger.

$\llbracket \varphi_1 \rrbracket^{\mathcal{K}} = \{2, 3, 5, 6\}\quad$ Die Knoten 3,5 und 6 erfüllen die Formel, da diese Knoten einen Nachfolger haben, der ein Terminalknoten ist.
Knoten 2 hat Knoten 5 als Nachfolger.
Deren Nachfolger haben entweder P-und Q-Nachfolger (3) oder gar keinen Nachfolger (7).\\

$\llbracket \neg\varphi_1 \rrbracket^{\mathcal{K}} = \{1,4,7\}\quad$ 1: Für 1 kommt nur Knoten 2 als Nachfolger infrage, der wiederum Knoten 5 als Nachfolger hat, welcher zwar ein P- aber keinen Q-Nachfolger hat.\\
4,7: 4 und 7 haben gar keine Nachfolger

\subsection*{b)}
$\varphi_2: $ ``Jeder P-Nachfolger jedes Nachfolgers von $v$ hat einen P-Nachfolger.''\\
$\llbracket \varphi_2 \rrbracket^{\mathcal{K}} = \{1, 4, 6, 7\}\quad$: 1: Der Knoten 1 erfüllt $\varphi_2$, da alle P-Nachfolger von 2 auch einen P-Nachfolger haben.\\
4,7: 4 und 7 erfüllen die Formel, weil es Terminalknoten sind.\\
6: 6 erfüllt die Formel, weil dessen einziger Nachfolger, 7, ein Terminalknoten ist.\\


$\neg\llbracket \varphi_2 \rrbracket^{\mathcal{K}} = \{2, 3, 5\}\quad$: 2 und 5 erfüllen die Formel nicht, da 3 4 als P-Nachfolger hat, 4 aber ein Terminalknoten ist.\\
3: 3 erfüllt die Formel nicht, da 6 7 als P-Nachfolger hat, 7 aber ein Terminalknoten ist.

\section*{Aufgabe 3}

\subsection*{a)}
\subsubsection*{(i)}
$\varphi := \square \square (P \rightarrow \lozenge P)$

\subsubsection*{(ii)}
$\varphi := \lozenge Q \rightarrow \lozenge P$

\subsubsection*{(iii)}
$\varphi := \square (\lozenge 1 \rightarrow \square ( \lozenge 1 \n \square \square 0))$

\subsection*{b)}
$\varphi := \lozenge (P \n Q) \n \lozenge(P \n \neg Q) \n \lozenge(\neg P \n Q) \n \lozenge(\neg P \n \neg Q)$

\end{document}
