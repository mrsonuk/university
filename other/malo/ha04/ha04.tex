\documentclass[11pt, a4paper]{article}

\usepackage[utf8]{inputenc}
\usepackage{pgffor}  %foreach
\usepackage{xstring} %string methods
\usepackage{fancyhdr}
\usepackage{ amssymb, amsmath, amsthm, dsfont }
\usepackage{ marvosym }
\usepackage{ stmaryrd }
%\usepackage[right = 1cm, top = 2cm]{geometry}
\usepackage[width = 18cm, top = 2cm, bottom = 3cm]{geometry}
\usepackage[pdf]{graphviz}
\usepackage{dot2texi}
\usepackage{tikz}
\usepackage{float}
\usetikzlibrary{shapes.geometric,shapes.misc}
\usepackage{tikz-qtree,tikz-qtree-compat}
%--- Own commands

\newcommand{\anf}[1]{``#1''}
\newcommand{\punkt}[2]{\hfill\begin{small} $[#1$ \textnormal{Punkt#2}$]$\end{small}}
\newcommand{\punkttabelle}[1]{$\hspace{0.5cm} /$ #1}
\newcommand{\abstractTabular}[3]
{
	\begin{center}
		\begin{normalsize}
			\begin{tabular}{#1}
				Aufgabe & #2 & $\Sigma$ \\
				\hline
				Punkte &  #3 \\
			\end{tabular}
		\end{normalsize}
	\end{center}
}

% -------------

\newcommand{\myTitleString}		{}
\newcommand{\myAuthorString}	{}
\newcommand{\mySubTitleString}	{}
\newcommand{\myDateString}		{}

\newcommand{\myTitle}		[1]{\renewcommand{\myTitleString}		{#1}}
\newcommand{\mySubTitle}	[1]{\renewcommand{\mySubTitleString}	{#1}}
\newcommand{\myAuthor}		[1]{\renewcommand{\myAuthorString}		{#1}}
\newcommand{\myDate}		[1]{\renewcommand{\myDateString}		{#1}}

\newcommand{\makeMyTitle}
{
	\thispagestyle{fancy} 					%eigener Seitenstil
	\fancyhf{} 								%alle Kopf- und Fußzeilenfelder bereinigen
	\fancyhead[L]							%Kopfzeile links
	{
		\begin{tabular}{l}
			\myTitleString
			\\ \mySubTitleString
			\\ \myDateString
		\end{tabular}
	}
	\fancyhead[C]
	{}	%zentrierte Kopfzeile
	\fancyhead[R]{\myAuthorString}	   		%Kopfzeile rechts
	\fancyfoot[C]{\thepage} 				%Seitennummer
	\text{}
}

% -------------

\renewcommand{\v}{\vee}
\newcommand{\n}{\wedge}
\newcommand{\xor}{\oplus}
\newcommand{\nmodels}{\nvDash}
\newcommand{\interp}{\mathfrak{I}}
\newlength\tindent
\setlength{\tindent}{\parindent}
\setlength{\parindent}{0pt}
\renewcommand{\indent}{\hspace*{\tindent}}

\begin{document}
\myTitle{MaLo}
\mySubTitle{Übung 04 - Gruppe A}
\myDate{SS 18}
\myAuthor
{
	\begin{tabular}{l l}
		381852, & Richard Zameitat \\
		378625, & Felix Dittmann \\
		380678, & Clemens Rüttermann \\
	\end{tabular}
}
\makeMyTitle

\section*{Aufgabe 1}
\subsection*{(a)}
$Q \Rightarrow Q, X, Z \v Y$\newline
----------------------------$(\v \Rightarrow)$\\
$\neg (Z \v Y), Q \Rightarrow Q, X \quad X, Q \Rightarrow Q,X \quad \neg (Z \v Y), Q \Rightarrow \neg Z, \neg Y \quad X, Q \Rightarrow \neg Z, \neg Y$\\
$(\v \Rightarrow)$----------------------------------------$\quad$------------------------------------------------------$(\v \Rightarrow )$\newline
$\neg (Z \v Y) \v X, Q \Rightarrow Q, X \qquad \neg (Z \v Y) \v X, Q \Rightarrow \neg Z, \neg Y$\\
$(\Rightarrow \v)$--------------------------$\qquad$---------------------------------------$(\Rightarrow \v)$\\
$\neg(Z \v Y) \v X, Q \Rightarrow Q \v X \quad \neg (Z \v Y) \v X, Q \Rightarrow \neg Z \v \neg Y$\\
------------------------------------------------------------------------$(\Rightarrow \n)$\\
$\neg (Z \v Y) \v X, Q \Rightarrow (Q \v X) \n (\neg Z \v \neg Y)$\\
------------------------------------------------------------------------$(\n \Rightarrow)$\\
$(\neg (X \v Y) \v X) \n Q \Rightarrow (Q \v X) \n (\neg Z \v \neg Y)$\\

Das Blatt $X,Q \Rightarrow \neg Z, \neg Y$ ist falsifizierend:\\
Falsifizierende Interpretation $\mathfrak{I}: Q \mapsto 1, X \mapsto 1, Y \mapsto 1, Z \mapsto 1$


\subsection*{(b)}
$A \Rightarrow C, B, A \quad A, B \rightarrow C, B$\\
$(\rightarrow \Rightarrow)$-----------------------------------------------\\
$A \rightarrow B, A \Rightarrow C, B, C$ \\
$(\Rightarrow \rightarrow)$------------------------------------------------------------------------\\
$A \rightarrow B \Rightarrow C,B,A \rightarrow C$\\
$(\neg \Rightarrow)$------------------------------------------------------------------------\\
$(A \rightarrow B), \neg(A \rightarrow C) \Rightarrow C,B$\\
$(\n \Rightarrow)$------------------------------------------------------------------------\\
$(A \rightarrow) \n \neg (A \rightarrow C) \Rightarrow C,B \quad\qquad C \Rightarrow C,B$ (Axiom) \\
------------------------------------------------------------------------$(\v \Rightarrow)$\\
$((A \rightarrow) \n \neg (A \rightarrow)) \v C \Rightarrow C, B$\\
------------------------------------------------------------------------$(\Rightarrow \v)$\\
$((A \rightarrow B) \n \neg (A \rightarrow C)) \v C \Rightarrow C \v B$\\

\section*{Aufgabe 2}
\subsection*{(a)}
Die Schlussregel $\frac{\Gamma, \varphi, \psi \Rightarrow \Delta \qquad \Gamma \Rightarrow \Delta, \varphi, \psi}{\Gamma \Rightarrow \varphi \oplus \psi}$, wobei $\oplus$ das exklusive Oder bezeichnet, ist korrekt.
Aus der rechten oberen Sequenz kann zunächst entnommen werdne, dass jedes Modell von $\Gamma$ entweder ein Modell von $\Delta, \varphi$ oder $\psi$ ist.
Wenn das Modell von $\Gamma$ auch ein Modell von $\Delta$ ist, sind wir fertig.
Sollte das Modell allerdings $\Delta$ nicht erfüllen, muss es einerseits entweder $\varphi$ oder $\psi$ erfüllen, wie wir (wie bereits erwähnt) der rechten oberen Sequenz entnehmen können, kann das Modell, wie wir der liken oberen Sequenz entnehmen können aber nicht beide erfüllen, da es sonst auch $\Delta$ erfüllt.
Somit ist die Schlussregel korrekt.


Korrektheit durch Ableitung:\\
Sei $\vartheta = (\neg \varphi \v \neg\psi) \n (\varphi \v \psi)$. Dann kann die Schlussregel $\frac{\Gamma, \varphi,\psi \Rightarrow \Delta \qquad \Gamma \Rightarrow \Delta, \varphi, \psi}{\Gamma \Rightarrow \Delta, \vartheta}$ folgendermaßen durch Ableitungen im Sequenzkalkül beweisen:\\


$\Gamma \Rightarrow \Delta, \vartheta$\\
------------------------------------------------------------------------(Def von $\vartheta$)\\
$\Gamma \Rightarrow \Delta, (\neg \varphi \v \neg \psi) \n (\varphi \v \psi)$ \\
------------------------------------------------------------------------$(\Rightarrow \n)$\\
$\Gamma \Rightarrow \Delta, \neg \varphi \v \neg \psi \qquad \Gamma \Rightarrow \Delta, \varphi \v \psi$\\
------------------------------------------------------------------------$(\Rightarrow \v)$\\
$\Gamma \Rightarrow \Delta, \neg \varphi, \neg \psi \qquad \Gamma \Rightarrow \Delta, \varphi, \psi$\\
------------------------------------------------------------------------$(\Rightarrow \neg)$\\
$\Gamma, \varphi, \psi \Rightarrow \Delta$\\


\subsection*{(b)}
Die Schlussregel $(\v \Rightarrow) \frac{\Gamma, \psi \Rightarrow \Delta \qquad \Gamma, \vartheta \Rightarrow \Delta}{\Gamma, \psi \v \vartheta \Rightarrow \Delta}$ ist korrekt, denn ein Modell von $\Gamma, \psi \v \vartheta$ erfüllt entweder $\Gamma$ und $\psi$, sodass wir aus der linken oberen Sequenz $\Gamma, \psi \Rightarrow \Delta$ entnehmen können,
dass es auch $\Delta$ erfüllt oder es erfüllt $\Gamma$ und $\vartheta$, so dass wir aus der rechten oberen Sequenz $\Gamma, \vartheta \Rightarrow \Delta$ entnehmen können, dass es auch $\Delta$ erfüllt.
Im Fall, dass das Modell $\Gamma, \psi$ und $\vartheta$ erfüllt, kann man aus beiden oberen Sequenzen erkennen, dass auch $\Delta$ erfüllt wird.

\subsection*{(c)}
Die Schlussregel $\frac{\Gamma \Rightarrow \Delta, \varphi \qquad \Gamma \Rightarrow \Delta, \psi}{\Gamma, \varphi \rightarrow \psi \Rightarrow \Delta}$ ist nicht korrekt.

Gegenbeispiel: $\Gamma = \{X \v Y \}, \Delta = \{X\}, \varphi = \psi = \{Y\}$\\
$X \v Y \Rightarrow X, Y$ ist korrekt, daher stimmen $\Gamma \Rightarrow \Delta, \varphi$ und $\Gamma \Rightarrow \Delta, \psi$\\
$X \v Y, Y \rightarrow Y \Rightarrow X$ ist allerdings nicht korrekt, da die Interpretation $\mathfrak{I}: X \mapsto 0, Y \mapsto 1$ zwar ein Modell von $X \v Y$ und $Y \rightarrow$ ist, aber nicht von $X$.


\section*{Aufgabe 3}

\underline{$(\Rightarrow |)$:}\\
\[ (\Rightarrow |) \, \frac{ \; \Gamma, \varphi, \psi \Rightarrow \Delta \; }{\Gamma \Rightarrow \varphi | \psi, \Delta}\]

Beweis der Korrektheit der Schlussregel durch Ableitung:
\renewcommand{\arraystretch}{.5}
\[\begin{array}{r c}
        & \Gamma, \varphi, \psi \Rightarrow \Delta \\
        (\n \Rightarrow) & \text{--------------------------} \\
        & \Gamma \varphi \n \psi \Rightarrow \Delta \\
        (\Rightarrow \neg) & \text{--------------------------} \\
        & \Gamma \Rightarrow \Delta, \neg (\varphi \n \psi) \\
        (\text{Definition von }|) & \text{--------------------------} \\
        & \Gamma \Rightarrow \Delta, \varphi | \psi
\end{array}\]
\\


$(| \Rightarrow ):$\\

$(| \Rightarrow) \frac{\Gamma \Rightarrow, \varphi \qquad \Gamma \Rightarrow \Delta, \psi}{\Gamma, \varphi | \psi  \Rightarrow \Delta}$\\

Beweis der Korrektheit der Schlussregel durch Ableitung per Sequenzkalkül:\\
$\Gamma \Rightarrow \Delta, \varphi \qquad \Gamma \Rightarrow \Delta, \psi$\\
--------------------------------$(\Rightarrow \n)$\\
$\Gamma \Rightarrow \Delta, \varphi \n \psi$\\
--------------------------------$(\neg \Rightarrow)$\\
$\Gamma, \neg (\varphi \n \psi) \Rightarrow \Delta$\\
--------------------------------Definition von $|$\\
$\Gamma, \varphi | \psi \Rightarrow \Delta$

\section*{Aufgabe 4}
\subsection*{(a)}
$(\mathbb{C}, +, \cdot) \quad (\mathbb{C}, +, <) \quad (\mathbb{C}, \cdot, <) \quad (\mathbb{C}, +) \quad (\mathbb{C}, \cdot) \quad (\mathbb{C}, <) \quad (\mathbb{C}, +, \cdot, <)$\\
$\{+, \cdot\}$-Redukt $\{+, <\}$-Redukt $\{\cdot, <\}$-Redukt $\{+\}$-Redukt $\{\cdot\}$-Redukt $\{+, \cdot, <\}$-Redukt

\subsection*{(b)}
\subsection*{i)}

\subsection*{ii)}
$\mathfrak{B} = $($\mathbb{Z}/5\mathbb{Z},+$ mod $5$) besitzt die folgenden Substrukturen:\newline
$\mathfrak{B_1} = $($\mathbb{Z}/5\mathbb{Z},+$ mod $5$) und\newline
$\mathfrak{B_2} = $($\{0\},+$ mod $5$).\newline
Die genannten Strukturen sind Substrukturen von $\mathfrak{B}$, da $\mathbb{Z}/5\mathbb{Z}, \{0\}\subseteq mathbb{Z}/5\mathbb{Z}$ gilt und die Universen $\mathbb{Z}/5\mathbb{Z}$ sowie $\{0\}$ abgeschlossen mit der Funktion $+$ mod $5$ abgeschlossen sind.

\subsection*{iii)}
$\mathfrak{C} = $($\mathbb{Z}/6\mathbb{Z},+$ mod $6$) besitzt die folgenden Substrukturen:\newline
$\mathfrak{C_1} = $($\mathbb{Z}/6\mathbb{Z},+$ mod $6$),\newline
$\mathfrak{C_2} = $($\{0\},+$ mod $6$),\newline
$\mathfrak{C_3} = $($\{0,3\},+$ mod $6$) und\newline
$\mathfrak{C_4} = $($\{0,2,4\},+$ mod $6$).\newline
Die genannten Strukturen sind Substrukturen von $\mathfrak{C}$, da die Universen Teilmengen des Universums von $\mathfrak{C}$ sind und die Universen außerdem abgeschlossen mit der Funktion $+$ mod $6$ abgeschlossen sind.


\end{document}
