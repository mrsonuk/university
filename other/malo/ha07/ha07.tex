\documentclass[11pt, a4paper]{article}

\usepackage[utf8]{inputenc}
\usepackage{pgffor}  %foreach
\usepackage{xstring} %string methods
\usepackage{fancyhdr}
\usepackage{ amssymb, amsmath, amsthm, dsfont }
\usepackage{ marvosym }
\usepackage{ stmaryrd }
%\usepackage[right = 1cm, top = 2cm]{geometry}
\usepackage[width = 18cm, top = 2cm, bottom = 3cm]{geometry}
\usepackage[pdf]{graphviz}
\usepackage{dot2texi}
\usepackage{tikz}
\usepackage{float}
\usetikzlibrary{shapes.geometric,shapes.misc}
\usetikzlibrary{arrows.meta,arrows}
\usepackage{tikz-qtree,tikz-qtree-compat}
\usepackage{forest}
\usepackage{pdflscape}
%--- Own commands

\newcommand{\anf}[1]{``#1''}
\newcommand{\punkt}[2]{\hfill\begin{small} $[#1$ \textnormal{Punkt#2}$]$\end{small}}
\newcommand{\punkttabelle}[1]{$\hspace{0.5cm} /$ #1}
\newcommand{\abstractTabular}[3]
{
	\begin{center}
		\begin{normalsize}
			\begin{tabular}{#1}
				Aufgabe & #2 & $\Sigma$ \\
				\hline
				Punkte &  #3 \\
			\end{tabular}
		\end{normalsize}
	\end{center}
}

% -------------

\newcommand{\myTitleString}		{}
\newcommand{\myAuthorString}	{}
\newcommand{\mySubTitleString}	{}
\newcommand{\myDateString}		{}

\newcommand{\myTitle}		[1]{\renewcommand{\myTitleString}		{#1}}
\newcommand{\mySubTitle}	[1]{\renewcommand{\mySubTitleString}	{#1}}
\newcommand{\myAuthor}		[1]{\renewcommand{\myAuthorString}		{#1}}
\newcommand{\myDate}		[1]{\renewcommand{\myDateString}		{#1}}

\newcommand{\makeMyTitle}
{
	\thispagestyle{fancy} 					%eigener Seitenstil
	\fancyhf{} 								%alle Kopf- und Fußzeilenfelder bereinigen
	\fancyhead[L]							%Kopfzeile links
	{
		\begin{tabular}{l}
			\myTitleString
			\\ \mySubTitleString
			\\ \myDateString
		\end{tabular}
	}
	\fancyhead[C]
	{}	%zentrierte Kopfzeile
	\fancyhead[R]{\myAuthorString}	   		%Kopfzeile rechts
	\fancyfoot[C]{\thepage} 				%Seitennummer
	\text{}
}

% -------------

\renewcommand{\v}{\vee}
\newcommand{\n}{\wedge}
\newcommand{\xor}{\oplus}
\newcommand{\nmodels}{\nvDash}
\newcommand{\interp}{\mathfrak{I}}
\newlength\tindent
\setlength{\tindent}{\parindent}
\setlength{\parindent}{0pt}
\renewcommand{\indent}{\hspace*{\tindent}}


\begin{document}
\myTitle{MaLo}
\mySubTitle{Übung 07- Gruppe A}
\myDate{SS 18}
\myAuthor
{
	\begin{tabular}{l l}
		381852, & Richard Zameitat \\
		378625, & Felix Dittmann \\
		380678, & Clemens Rüttermann \\
	\end{tabular}
}
\makeMyTitle

\section*{Aufgabe 1}
\subsection*{a)}
Angenommen $\mathfrak{A}$ wäre nicht starr. Dann existiert ein $\pi: A \to A$, für welches ein $a\in A$ existiert, sodass $\pi(a) \neq a$ gilt und $\mathfrak{A} \vDash \varphi(a_1,...,a_n)$ genau dann, wenn $\mathfrak{A} \vDash \varphi (\pi (a_1),..., \pi(a_n))$. Da jedes Element von $A$ allerdings elementar definierbar ist, gibt es eine Definition von $a$, sodass $\mathfrak{A} \vDash \varphi_a (a)$ und $\mathfrak{A} \nVDash \varphi_a (\pi(a))$. Dies widerspricht unserer Annahme, sodass $\mathfrak{A}$ starr sein muss.

\subsection*{b)}
\subsection*{i)}
$\mathfrak{Q} = (\mathbb{Q}, + , \cdot)$ ist starr. Dies liegt daran, dass $\mathbb{Q}$ in der Struktur elementar definierbar ist. Eine beliebige Zahl $q\in \mathbb{Q}$ lässt sich darstellen durch $q = a/b, a,b\in \mathfrak{Z}$. Da wir in der Struktur allerdings nur die Multiplikation haben, müssen wir hier auf die Darstellung $q * b =  a$ zurückgreifen. Nun müssen wir also nur noch zeigen, dass auch $\mathbb{Z}$ elementar definierbar ist. Dazu definieren wir zunächst die $1$. Diese können wir mit der Formel $\varphi_1(x) = \forall y (x \cdot y = y)$ machen, da $1$ das neutrale Element der Multiplikation ist. Die $0$ können wir auch direkt als neutrales Element der Addition mit der Formel $\varphi_0 (x) := \forall y (y+x = y)$ definieren.\newline
Mit der $1$ und der Addition können wir nun jedes $n\in \mathbb{N}$ mit einer entsprechenden Formel $\varphi_n(x) = (x = \varphi_{n-1} + 1)$ definieren. Um die negativen Zahlen definieren zu können, benötigen wir die $-1$. Diese definieren wir mit der Formel $\varphi_{-1} = \exists y (x \cdot x = y \land \varphi_1(y))\land (\neg \varphi_1(x))$. Die negativen Zahlen können wir somit mit der Formel $\varphi_{-n}(x) = \exists y \exists z (\varphi_{-1}(y) \land \varphi_n(z) \land (z \cdot y) = x)$ definieren.\newline
Somit haben wir gezeigt, dass $\mathbb{Z}$ definierbar ist, sodass wir nun auch jedes beliebige $q \in \mathbb{Q}$ durch die Formel $\varphi_\mathbb{q}(x) := \exists a \exists b (b \cdot x = a \land \varphi_n(b) \land \varphi_z(a))$ definieren, wobei $\varphi_z$ die Formel von $\varphi_n$ oder $\varphi_{-n}$ bezeichnet, die hier benötigt wird. Da jedes Element aus dem Universum von $\mathfrak{Q}$ definierbar ist, ist $\mathfrak{Q}$ starr.

\subsection*{ii)}
$\mathfrak{R}$ ist nicht starr. Gegenbeispiel: Für $\mathfrak{R}$ existiert ein Automorphismus $\pi: \mathbb{R} \to \mathbb{R}$ mit $\pi_-(a) = -a$. Für $\pi_-$ gilt $\pi_-(a+b) = \pi_-(a) + \pi_- (b)$:\newline
$\pi_-(a) + \pi_-(a) = (-a) + (-b) = (-1)*a + (-1)*b = (-1)*(a + b) = \pi_-(a + b)$. Da $\tau = \{+\}$ ist, gilt somit, dass wenn $\mathfrak{R} \vDash \varphi(a_1, ..., a_k)$ gilt, auch $\mathfrak{R} \vDash \varphi(\pi_-(a_1), ..., \pi_-(a_k))$ gilt.

\subsection*{iii)}
Die Knoten des Graphen sind elementar definierbar:\newline
$\varphi_1(x): \forall y (\neg Eyx)$\newline
$\varphi_4(x): \forall y (\neg Exy)$\newline
$\varphi_3(x): \exists y \exists x (\varphi_1(y) \land \varphi_4(z) \land Eyx \land Exz)$\newline
$\varphi_2(x): \exists y \exists z (\varphi_1(y) \land \varphi_3(z) \land Eyx \land Exz)$\newline
Wie in a) gezeigt, ist $\mathfrak{G}$ somit starr.

\subsection*{iv)}
$\mathfrak{A}$ ist nicht starr. Gegenbeispiel: Es existiert ein Automorphismus $\pi_+: \mathbb{Z} \to \mathbb{Z}$ mit $\pi_+(a) = a+30$. Zunächst gilt, wenn $a<b$ gilt, gilt auch $a+30<b+30$. Des Weiteren gilt, dass wenn $a \in 5\mathbb{Z}$ gilt, auch $a+30 \in 5\mathbb{Z}$ gilt, da $30 mod 5 = 0$, und schließlich, wenn $a\in 6\mathbb{Z}$, dann auch $a+30 \in 6\mathbb{Z}$. Somit wurde für jede Relation $R$ aus $\mathfrak{A}$ gezeigt, dass wenn $a\in R$ gilt, auch $\pi_+(a) \in R$ gilt.


\section*{Aufgabe 2}
\subsection*{a)}
\[
    \pi: x=\left(\frac{a}{b}\right) \mapsto \left\{
        \begin{array}{lr}
            x & \quad \text{ für } x = 0 \\
            \frac{b}{a} & \quad \text{sonst}
        \end{array}
    \right.
\]

$\pi$ ist ein Automorphismus auf $(\mathbb{Q}, \cdot)$.
Dieser bildet, beispielsweise $\frac{2}{1} \in \mathbb{Q}_{\geq 1}$ auf $\frac{1}{2} \notin \mathbb{Q}_{\geq 1}$ ab.
Somit ist ein Gegenbeispiel gefunden und daher $\mathbb{Q}_{\geq 1}$ nicht elementar auf $(\mathbb{Q}, \cdot)$.

\subsection*{b)}
$\varphi(x) := \exists y (y + y + y = x)$

\subsection*{c)}
$\varphi_1(x) := \forall y (x \cdot y = y)$\\
Sei $\varphi_{<}(x)$ die in der Vorlesung definierte kleiner Relation.\\
$\varphi_{ggt}(a,b) := \exists g (\exists x \; (x \cdot g = a \rightarrow x < g \n \neg \exists z \; \exists y (g < z \n g \cdot z = a))
\n (\exists x \; (x \cdot g = b \rightarrow x < g \n \neg \exists z \; \exists z \; \exists y (g < z \n g \cdot z = b) )) )$


\subsection*{d)}
$\varphi_0(x) := \forall y \; x \neq y \rightarrow x < y$\\
$\varphi_1(x) := \exists n \; \varphi_0(n) \n n \neq x \n \forall y \; (x \neq y \n n \neq y \rightarrow x < y)$\\
$\varphi_2(x) := \exists n \; \exists e \; \varphi_0(n) \n \varphi_1(e) \n n \neq x \n e \neq x \forall y \; (x \neq y \n n \neq y \n e \neq y \rightarrow x < y)$\\

$\varphi_{\{0,1,2\}}(x) := \exists n \;(\varphi_0(n) \n x = n) \v \exists e \;(\varphi_1(e) \n x = e) \v \exists z \;(\varphi_2(z) \n x = z)$

\subsection*{e)}
%$\varphi_{diff_1}(a,b) := ((a < b) \rightarrow \neg \exists x (a < x \n x < b)) \v ((b < a) \rightarrow \neg \exists x (b < x \ x < a))$\\
%$\varphi_{diff_2}(a,b) := ((a < b) \rightarrow \exists x (\varphi_{diff_1}(a,x) \n \varphi_{diff_1}(x, b))) \v ((b < a) \rightarrow \exists x (\varphi_{diff_1}(b,x) \n \varphi_{diff_1}(x,a)))$\\


Vorschlag 2:


\[ \pi: q \mapsto 2*q \]

$\pi$ ist ein Automorphismus auf $(\mathbb{Q}, <)$.
Für $a := 4, b := 1$ gilt $|a - b| = |4 - 1| = |3| = 3$.
Jedoch ist $|\pi(a) - \pi(b)| = |8 - 2| = |6| = 6 \neq 3$.
Somit ist ein Gegenbeispiel gefunden und deshalb $\{(a,b) : |a-b| = 3\}$ nicht elementar definierbar in $(\mathbb{Q},<)$.

------------------------------

$\varphi_{diff_1}(a,b) := ((a < b) \rightarrow \neg \exists x (a < x \n x < b)) \v ((b < a) \rightarrow \neg \exists x (b < x \ x < a))$\\
$\varphi_{diff_2}(a,b) := ((a < b) \rightarrow \exists x (\varphi_{diff_1}(a,x) \n \varphi_{diff_1}(x, b))) \v ((b < a) \rightarrow \exists x (\varphi_{diff_1}(b,x) \n \varphi_{diff_1}(x,a)))$\\

%$\varphi(a,b) := ((a < b) \rightarrow \exists x (\varphi_{diff_1}(a,x) \n \varphi_{diff_2}(x, b))) \v ((b < a) \rightarrow \exists x (\varphi_{diff_1}(b,x) \n \varphi_{diff_2}(x, a)))$

\section*{Aufgabe 3}
\subsection*{a)}
\subsubsection*{i)}
Die Aussage, wenn $\mathfrak{A} \subseteq \mathfrak{B} \preceq \mathfrak{C}$ und $\mathfrak{A} \preceq \mathfrak{C}$ gilt, dann auch $\mathfrak{A} \preceq \mathfrak{B}$, ist korrekt. Da sowohl $\mathfrak{A} \preceq \mathfrak{C}$ als auch $\mathfrak{B} \preceq \mathfrak{C}$ gilt, bedeutet dies, dass sowohl $\mathfrak{B} \vDash \varphi(x_1,...,x_k)$ als auch $\mathfrak{A} \vDash \varphi(x_1,...,x_k)$ genau dann gelten, wenn $\mathfrak{C} \vDash \varphi(x_1,...,x_k)$ gilt. Dadurch gilt auch, dass $\mathfrak{A} \vDash \varphi(x_1,...,x_k)$ genau dann gilt, wenn $\mathfrak{C} \vDash \varphi(x_1,...,x_k)$ gilt, und da $\mathfrak{A} \subseteq \mathfrak{B}$ ist, gilt schließlich auch $\mathfrak{A} \preceq \mathfrak{B}$.

\subsubsection*{ii)}

\subsection*{b)}

\subsection*{c)}
"$\Rightarrow$": Wenn $\mathfrak{A} \preceq \mathfrak{B}$ gilt, gilt auch, dass wenn $\mathfrak{B} \vDash \exists y \varphi (a_1,...,a_k,y)$ mit $a_1,...,a_k \in A$ gilt, auch direkt $\mathfrak{B} \vDash \varphi(a_1,...,a_k,a)$ für ein $a\in A$ gilt.\newline
Nehmen wir zunächst an, dass die Behauptung nicht stimmt. Dann existiert kein $y$ in $\varphi$, sodass $\mathfrak{B} \vDash \varphi(a_1,...,a_k,y)$ mit $y\in A$ gilt. Dies würde allerdings dazu führen, dass $\mathfrak{A} \nvDash \varphi(a_1,...,a_k)$ gilt, da A das Universum von $\mathfrak{A}$ ist. Dies ist ein Widerspruch der Definition elementarer Substrukturen, da dann $\mathfrak{B} \vDash \varphi(a_1,...,a_k)$ und $\mathfrak{A} \nvDash \varphi(a_1,...,a_k)$ gilt.\newline\newline
"$\Leftarrow$": Wenn die Implikation, wenn $\mathfrak{B} \vDash \exists y \varphi(a_1,...,a_k,y)$ mit $a_1,...,a_k \in A$ gilt, gilt auch direkt $\mathfrak{B} \vDash \varphi(a_1,...,a_k,a)$ für ein $a\in A$, gilt, gilt auch $\mathfrak{A} \preceq \mathfrak{B}$.\newline
Hierbei unterscheiden wir zwei Fälle: $y\notin frei(\varphi)$ und $y \in frei(\varphi)$.\newline
$y \notin frei(\varphi)$: Die zusätzliche Variable hat keinen Einfluss auf $\phi$. Somit gilt $\mathfrak{B} \vDash \varphi(a_1,...,a_k)$. Da $a_1,...,a_k \in A$ gilt, gilt auch $\mathfrak{A} \vDash \varphi (a_1,...,a_k)$, da $\mathfrak{B}$ nur Elemente aus $A$ verwendet hat.\newline
$y \in frei(\varphi)$: Dieser Fall gilt analog zu $y \notin frei(\varphi)$, da es aufgrund unserer Voraussetzung ein $a \in A$ gibt, welches für $y$ verwendet werden kann. Somit gilt für alle FO($\tau$)-Formeln, wobei $\tau$ die Signatur von $\mathfrak{B}$ ist, $\mathfrak{B} \vDash \varphi(x_1,...,x_k,y)$ genau dann, wenn $\mathfrak{A} \vDash \varphi(x_1,...,x_k,y)$, sodass $\mathfrak{A} \preceq \mathfrak{B}$ gilt.

\section*{Aufgabe 4}
\subsection*{a)}
$\simeq$ ist dann eine Äquivalenzrelation, wenn sie reflexiv, symmetrisch und transitiv ist.

\subsubsection*{Reflexivität:}
Reflexivität gilt genau dann wenn $a \simeq a \forall a \in A$ gilt.
Dafür muss einen Automorphismus $\pi$ existieren mit $\pi(a) = a$.
Da die Identität ein solcher Automorphismus ist ist $\simeq$ reflexiv.

\subsubsection*{Symmetrie:}
Sei $a \simeq b$ also existiert ein Automorphismus $\pi$ für $a$ und $b$ mit $\pi(a) = b$.
Da Automorphismen bijektiv sein müssen, existiert eine Umkehrfunktion $\pi^{-1}$.
Somit ist $\pi^{-1}(b) = a$ und somit ist $\simeq$ symmetrisch.


\subsubsection*{Transitivität:}
Sei $a \simeq b$ und $b \simeq c$, d.h. es existieren die Automorphismen $\pi_1(a) = b$ und $\pi_2(b) = c$.
Dann muss es aber auch den Automorphismus $\pi_1(\pi_2(a)) = c$ wesshalb auch $a \simeq c$ gilt.


\subsection*{b)}

Für jedes Paar von Elementen einer Äquivalenzklasse gilt, dass es einen Automorphismus gibt, der das eine Element auf das andere Abbildet (siehe Teil a)).
Da es einen Automorphismus gibt, der die Elemente vertauscht, sind diese selbst nicht mehr elementar definierbar.
Somit sind höchstens noch die Mengen, welche zu den Äquivalenzklassen gehören, selbst elementar definierbar.
Also gibt es (maximal) $2^k$ elementar definierbare Teilmengen von $A$, da für jede Menge einer Äquivalenzklasse entschieden werden kann, ob sie in die Teilmenge aufgenommen wird, oder nicht.



\section*{Aufgabe 5}

\subsection*{a)}


\subsection*{b)}

\subsection*{c)}

\subsection*{d)}

\end{document}
